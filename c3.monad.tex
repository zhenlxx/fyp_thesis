\chapter{Monad in Haskell}
\section{Haskell and Category Theory}
Category theory is a general theory that examine and organize mathematical object like set ,function,function domains Cartesian-set.

A Category $C $ in category theory is defined below :
\begin{enumerate}
\item a collection of objects 
\item a collection of arrows (often call morphism) 
\item operations assigning to each arrow $f$ an object $dom\;f$,its domain ,and an object $cod\;f$,its co domain.
\item a composition operator assigning to each pair of arrows $f and g$,with $cod\;f = dom\;g$,a composite arrow $ g \circ f:dom\;f \rightarrow  cod\;g$ , satisfying the following associative law: \\
For any arrow $f: A \rightarrow B,g:B \rightarrow C,and\;h: C\rightarrow D$(with A,B,C and D not necessarily distinct),
$$h\circ (g\circ f) = (h\circ g)\circ f$$
\item for each object A, an identify arrow $id_{a}: A \rightarrow A$ satisfying the following identity law:\\
For any arrow $ f: A \rightarrow B,$ 
$$ id_{a} \circ f = f  \;and\;  f\circ id_{a} = f. $$
\end{enumerate}\cite{pierce_basic_1991}

Functions are the first member of the program in functional programming,since no size affect is not allow ,there should be a way to combine the all kinds of functions to from a new function instead of just simply chain the input output of each function as the former will generate intermediate output.


For instance ,counting the file of java source code in current directory can be written as follow:


	$$ ls-al . | grep *.txt| wc -l $$ 
	
	
To substantiate the this concept , let's use the map/fold fusion technique of Haskell as an example.

If we want to calculate the sum of the square of each element of a list eg. [1,3,4,6,7,9],the result of it is  $ 1^2+3^2+4^2+6^2+7^2+9^2=192 $.In Haskell ,we could use map and fold to address problem.

 %list_of_square = map (^2)
%sum_of_list = foldr (+) 0
%sum_of_square =   sum_of_list.list_of_square

To avoid generating intermediate output from the first function to second function, the could rewite the hold function using a single fold

The all map/fusion is is equivalent to 
$ foldr f e . map g = foldr (\\x y -> f (g x) y) e $

therefore, the 

$ sum_of_square = foldr (\\x y -> x^2 + y) 0 $





\section{Monadic Function}
\section{Using Monad Operator to Combine Monadic Function}

\section{Type system in Haskell}
\section{Do Notation - Reinventing a imperatilanguageve language}

