\chapter{Introduction}
\section{Objective and Motivation}
The objective of this project is to develop a dynamic and weak typing interpreted language using Haskell.This language is able to support the following features,
\begin{itemize}
\item basic for loop and while loop
\item basic if-else statement
\item functional invocation 
\item arbitrary dimension list
\item polymorphic list
\end{itemize}

Furthermore, in this project,the monadic design approach is applied as Haskell is different from other object oriented language.

\section{Benefits of using Haskell} 
Haskell is an advanced purely-functional programming language.By applying the used of Haskell to this project ,I have significantly reduce the coding time and spent most of my time to the design phrase.
\\
\\
A pure function is a function that accepts an input and produce an output.In Object-Oriented language,program is constructed using classes and instances which encapsulate computations and states.Haskell program is constructed by functions as function is the first class member in Haskell.Typically the main function is defined in terms of other functions, which in turn are defined in terms of still more functions, until at the bottom level the functions are language primitives. All of these functions are much like ordinary mathematical functions.
[Why Functional Programming Matters]
[Functional Programming with Overloading and
Higher-Order Polymorphism
]
\section{Development Methodology}
Agile development methodology is used in the entire development process.This project has been initially identified to multiple iterations and each iteration contains three major phrases inducing research ,development and testing.
\\
\\
Researches have been done on investing some useful Haskell library like Parsec ,Happy,Alex and HUnit.
\\
\\
Eclipse with EclipseFP plugin provide support for source code and package management in Eclipse.To keep tract of the development process,I make use of \textbf{git} a source control tool to version all the source file.



\section{Code Formats}
In this thesis,two types of code will be listed in different formats.

Code listed without line numbers illustrates the design of the interpreter or other Haskell concept.Most of it are Haskell code,which looks like,
\begin{hcode}
 return a >>= k  ==  k a
 m >>= return  ==  m
 m >>= (\x -> k x >>= h)  ==  (m >>= k) >>= h
\end{hcode}


Code listed with line numbers are example of \textbf{yun} ,the programming language designed in this project.It looks like,

\begin{lstlisting}[language=SQL]
program main (){
	result = fib (10);
	sys.printLn(result);	
} 
function fib (num) {
	if (num ==0){
		return 1;
	}
	if(num == 1){
		return 1;
	}
	
	part1 = fib(num-1);
	part2 = fib(num-2);
	return (part1+part2) ;
}
\end{lstlisting}