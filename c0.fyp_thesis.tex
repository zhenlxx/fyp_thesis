%=========================================================================
% This is a template for a Final Year project, which is essentially a 
% document of type "report" with a modified title page
% To create a pdf compile this file using PDFLaTeX. In Texmaker this is 
% the PDFLAT button on the toolbar. You don't need to complie any 
% intermediate formats such as .dvi.
%=========================================================================

%=========================================================================
% This section sets up the standard layout and includes necessary latex 
% packages.
%-------------------------------------------------------------------------
\documentclass[11pt,a4paper]{report}
\usepackage[utf8x]{inputenc}
\usepackage{ucs}
\usepackage{amsmath}
\usepackage{amsfonts}
\usepackage{amssymb}
\usepackage{makeidx}
\usepackage{graphicx}
\usepackage{listings} 
\usepackage[gen]{eurosym}          
\usepackage{url}
\usepackage{fancyvrb}
\DefineVerbatimEnvironment{hcode}{Verbatim}{fontsize=\small}
\DefineVerbatimEnvironment{hexample}{Verbatim}{fontsize=\small}
\newcommand{\ignore}[1]{}


\begin{document}

%=========================================================================
% In
%-------------------------------------------------------------------------
%=========================================================================
% This contains basic project information (Title/Author/Supervisor etc.)
% which should be filled in by the student.
%-------------------------------------------------------------------------
\newcommand{\inProjectTitle}{Implmenting a interpreter for a scripting lanaguge using Haskell}
\newcommand{\inInterimReport}{interim report}
\newcommand{\inAuthorFName}{Zhen}
\newcommand{\inAuthorLName}{Lao}
\newcommand{\inAuthorEmail}{zhen.lao@student.dit.ie}
\newcommand{\inSupervisorFName}{Richard}
\newcommand{\inSupervisorLName}{Lawlor}
\newcommand{\inSecondReaderFName}{Cindy}
\newcommand{\inSecondReaderLName}{Liu}			% student edits this - title, name, email etc.
%=========================================================================
% This lays out the title page. 
% DO NOT EDIT THIS PAGE except to uncomment the text below -  
% "This Report is submitted..." to reflect the programme title.
%-------------------------------------------------------------------------

% Commands to make entry and formatting of Title/Author info more intuitive
\newcommand\ProjectTitle[1]{\begin{LARGE}\textsc{#1}\end{LARGE}\\[1.25cm]}
\newcommand\InterimReport[1]{\begin{LARGE}\textsc{#1}\end{LARGE}\\[1.25cm]}
\newcommand\AuthorName[2]
{\begin{LARGE}#1 \textsc{#2}\end{LARGE}\\[0.25cm]}
\newcommand\AuthorEmail[1]{\begin{Large}\texttt{#1}\end{Large}\\[0.75cm]}
\newcommand\Supervisor[2]{\begin{large}\textit{Supervisor: }#1 \textsc{#2}\end{large}\\[0.25cm]}
\newcommand\SecondReader[2]{\begin{large}\textit{2nd Reader: }#1 \textsc{#2}\end{large}}

\begin{titlepage}
\begin{center}

% Upper part of the page contains DIT logo disable the dit logo
\includegraphics[scale=0.60]{pic/s1.png}\\[2.5cm]
 
% Title 
\ProjectTitle{\inProjectTitle}
\InterimReport{\inInterimReport}
% Author and supervisor
\AuthorName{\inAuthorFName}{\inAuthorLName} 
\AuthorEmail{\inAuthorEmail}
\Supervisor{\inSupervisorFName}{\inSupervisorLName}
\SecondReader{\inSecondReaderFName}{\inSecondReaderLName}

% Bottom of the page
\vfill
{\large \today}\\[0.5cm]  % Date compiled

This Report is submitted in partial fulfillment of the requirements for the award of the degree of
\textbf{BSc Computer Science}  % DT228
% \textbf{BSc Computing}  % DT211
of the School of Computing, College of Sciences and Health, Dublin Institute of Technology.
\end{center}
\end{titlepage}			% no editing required here

% Abstract 
\begin{abstract}

\begin{flushleft}
\textbf{Keywords:}programming language 
\textsc{yun}
\end{flushleft}
\end{abstract} 
%=========================================================================
% This contains a declaration that the student has complied with the rules  
% regarding plagairism. The paper copy of this page must be signed by the 
% student and supervisor before submission.
%-------------------------------------------------------------------------
\section*{Declaration}
I \textbf{\inAuthorFName} \textbf{\inAuthorLName} hereby declare that the work described in this dissertation is, except where otherwise stated, entirely my own work and has not been submitted as an exercise for a degree
at this or any other university.


\vfill

\begin{tabular}{ l p{4cm} }
Signed &  \\ \cline{2-2}
 & \textit{\inAuthorFName}  \textit{\inAuthorLName} \\ [2.0cm]
\end{tabular}

\pagebreak		% needed because next (probably Acknowledgements) is a section not a chapter			% needed only if the document must have a Declaration
% Acknowledgements
\section*{Acknowledgements}
I would like to thank my supervisor Richard Lawor,for his valuable advice and useful suggestions on my project.\\
I am also deeply indebted to all the other tutors and teachers in Computer Science for their direct and indirect help to me.\\
Special thanks should go to my friends who have put considerable time and effort into their comments on the draft. 

%=========================================================================
% Use the next 3 commands to include Abstratct/Declaration/Acknowlegdements in 
% the table of contents 
%-------------------------------------------------------------------------
%\addcontentsline{toc}{section}{Abstract}
%\addcontentsline{toc}{section}{Declaration}
%\addcontentsline{toc}{section}{Acknowledgements}

%=========================================================================
% Autogenerate tables of contents, figures and tables
%-------------------------------------------------------------------------
\tableofcontents
\listoffigures
%\listoftables
%\lstlistoflistings

%=========================================================================
%  Set up environment for code listings
%-------------------------------------------------------------------------
\lstset{
	basicstyle=\small,          			% print whole listing small
	keywordstyle=\bfseries\underbar,		% bold underline keywords
	stringstyle=\ttfamily,      			% typewriter type for strings
	identifierstyle=\bfseries,				% bold identifiers
	showstringspaces=false,     			% no special string spaces
 	numbers=left, 
	numberstyle=\tiny, 
	stepnumber=1, numbersep=5pt,
	breaklines,
	breakatwhitespace	
	}

%=========================================================================
% Actual chapters start here.
% 
% The basic idea is that each chapter is a separate .tex file which is 
% included at compile time. Each file should have a meaningful filename.
%-------------------------------------------------------------------------
\chapter{Introduction}
\section{Objective and Motivation}
The objective of this project is to develop a weak-type interpreted language using Haskell.This language is able to support the following feature,
\begin{itemize}
\item basic for loop and while loop
\item basic if-else statement
\item functional invocation 
\item arbitrary dimension list
\item polymorphic list
\end{itemize}

Furthermore, in project,the monadic design approach is applied as Haskell is different from other object oriented language.

\section{Benefits of using Haskell} 
Haskell is an advanced purely-functional programming language.By applying the used of Haskell to this project ,I have significantly reduce the coding time and spent most of my time to the design phrase.
\\
\\
A pure function is a function that accept an input and generate  an output.In Object-Oriented language,program are constructed using class and instance which encapsulate computation and state.Haskell program is construct by function as function is the first class member in Haskell.Typically the main function is defined in terms of
other functions, which in turn are defined in terms of still more functions, until at the bottom level the functions are language primitives. All of these functions are much like ordinary mathematical functions.
[Why Functional Programming Matters]
[Functional Programming with Overloading and
Higher-Order Polymorphism
]
\section{Development Methodology}
Agile development methodology is used in the entire development process.This project has been initially identified to multiple iterations and each iteration contains three major stages incuding research ,development and testing.
\input{c2.tech.tex}
\chapter{Future Work and Project Plan}
\section{Future Work}
I have done most of the research work of the project.the future work will be implementing the actual parser.There will be an initial implementation of part of the EBNF definition.The execution engine will be implemented at the same as the parser.Due to Haskell have a powerful abstraction mechanism and a suitable library (Parsec),the implementation will not be too difficult.

The major barrier will be implementing the execution engine since I have no knowledge about it.
There will be more research in the later stage of this project on the interpreter part as well as the grammar and Haskell part.

\section{Project Plan}
\begin{tabular}{|p{1.00\textwidth}|}

\hline \textbf{Before November} \\
I have finish most of research on Chomsky's 
CFG grammar and Haskell.I have started to implement a prototype of the interpreter.\\

\hline \textbf{November 12 to November 26}\\
Research on the execution engine of the interpreter.\\
\hline \textbf{November 12 to November 31} \\
Development a prototype by following the documentation of parsec. The document  "Write Yourself a Scheme in 48 Hours/Parsing" offer an example to implements a interpreter for Scheme using parsec library of Haskell.\\
\hline
\textbf{November 31 to January 15} \\
Implement a subset of EBNF specification of $yun$.Add the error checking to the interpreter.Meantime,as the code growing ,unit test will be added to guarantee the quality of existing code.
\\
\hline  \textbf{January 16 to February 15}\\
Implement all the EBNF specification of $yun$\\
\hline   \textbf{February 31 to March 15}\\
Implement the IO command (library).Add more test code.\\
\hline \textbf{February 16 to March 16}\\
Review the EBNF of $yun$ programming language.Implement more library for it.Start the system testing.
\\ 
\hline \textbf{March 17 to April 8}\\
Prepare for the project fair.\\
\hline
\end{tabular} 









%=========================================================================
% This section renders the bibliography
%-------------------------------------------------------------------------
\clearpage										% these two lines needed to add Bibliography
\addcontentsline{toc}{chapter}{Bibliography}   	% to Table of Contents
\bibliographystyle{plain}
\bibliography{Biblio1} 		% You can include multiple files. 
							% There is no whitespace between the commas and the next bib file.


%=========================================================================
% Everything after here is an appendix
%-------------------------------------------------------------------------
\appendix	

%\input{References.tex}
%\input{Glossary.tex}
%\chapter{Sample appendix title}

\section{Section title}


% Example of how to include code. "MainWebPage.html" (in this example) should be
% replaced with a real file name. The language parameter may also have to be changed.

%\lstinputlisting[language=XML, caption=Main Web Page]{./MainWebPage.html}


\end{document}