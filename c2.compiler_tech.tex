\chapter{Compiler and Interpreter Technologies}
\section{Parsing technologies}

\subsection{The Hierarchy of Grammars}
Noam Chomsky has describe three model of grammar ["Three models for the description of language"]  and this grammar model has significantly effect the design of computer programming language.


Chomsky define a set of rule upon the formal grammar and categorize them into different levels.

A formal grammar of this type consists of:
\begin{itemize}
\item a finite set of terminal symbols.
\item a finite set of non-terminal symbols.
\item a finite set of project rules.
\item a start symbol.
\end{itemize} \cite{aho1986compilers}

From previous formal grammar definition, legitimate production rules can be written as 
 \[ S \mapsto aS  \,and \, S \mapsto ab \]
 
In this example,we can assume that the grammar consists of two projection rules and the starting symbol is $ S $.The terminal symbols are lower letters $ \{a ,b\} $ . From this example, If we start from the either rule 1 or rule 2 ,we could derive a  grammar of $ \{ a^n b | n>1  \}$ ,which can be enumerate like $ \{aab,aaab,aaaab,\cdots \} $.

The Chomsky hierarchy consists of the 4 levels:
\begin{itemize}
\item Type-0 grammars. It is a unrestricted grammars that include all.
\item Type-1 grammars.
\item Type-2 grammars.
\item Type-3 grammars.
\end{itemize}

\subsection {Backus–Naur Form and Extended Backus–Naur Form}
The Backus-Naur Form(BNF) is a metalanguage to write the production rule that expressing the type-2 grammar (context-free grammar).It  restricts the appearance of terminal and non-terminal in each side of the production equation.A canonical BNF production rule may like follow,
 \[   <symbol> ::= \_\_expression\_\_ \]
 
The left side of the equation can only be non-terminal thus enclosed with $<>$ .The right hand side can be terminals and non-terminals,a vertical bar '|' is used to represent choice between terminal and non-terminals.\\

The  Extended Backus–Naur Form (EBNF) and extension upon the BNF.Three regular expression qualifier is added to simplified some expression,they are,
\begin{itemize}
\item ? : which means that the symbol (or group of symbols in parenthesis) to the left of the operator is
optional (it can appear zero or one times)
\item * : which means that something can be repeated any number of times (and possibly be skipped
altogether)
\item + : which means that something can appear one or more times 
\end{itemize} \cite{book}

Recrusive rules of BNF  like 
\[   1. <exp> := <exp> | sub \]
\[	 2. <exp> := sub     \]
 
that expressing a sequence of a particular syntactic element can be simplified using quantifier in EBNF as $ <exp>:=sub+ $



\section{Parser Generator Haskell Happy}
Happy is a parser generator system for Haskell, similar to the tool `yacc' for C. Like `yacc', it takes a file containing an annotated BNF specification of a grammar and produces a Haskell module containing a parser for the grammar.
[The Parser Generator for Haskell] 

By using its own EBNF like syntax,used could write an parser description.The happy parser generator are able to recognize and compile it into Haskell source code.


\section{Monadic Parsing using Parsec}
In the early,stage of this project,parse is build using parse C,
Parsec is an industrial strength, monadic parser combinator library for Haskell. It can parse context-sensitive, infinite look-ahead grammars but it performs best on
predictive (LL[ Compilers: principles, techniques and tools.]) grammars. Combinator parsing is well known in the literature
and offers several advantages to YACC or event-based parsing. [Parsec, a fast combinator parser] 

Compared toe parser generator , monadic parsing has two major benifits
1. No need to learn additional parser generator grammar since parser combinator is written in the same language. 2.parser can be adjust easily .






\section {Lexical analysis}
Before parsing,the lexical analsier will scan the source code and generate a sequence of token.


\subsection{Regular Expression and token}
Tokens are defined by using regular expression,

\subsection{The Lexer Generator Alex}
In this project, the Alex Haskell Lexer generator is apply in generating token streams.

each token can be defined using regular expression.



\section{Language definition}

Production rule in EBNF 


