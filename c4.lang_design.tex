\chapter{Language Interpreter Design}

\section{Type System}


\subsection{Dynamic Typing and Static Typing}
A programming language is said to use static typing when type checking is performed during compile-time as opposed to run-time. 
For example , you have to specify the type explicitly and the compile will the the type correctness of a variable.A variable of a specified type can not assign to another value of other type.

\begin{tabular}{p{5cm}|p{5cm}}
\hline
Static Typing & Dynamic Typing \\
\hline
int a =1;\par \textbf{/*a is of type int */} \par int a="a string"; \par \textbf{ /* its not valid to assign a string to a variable that has type int */} & a=1; \par \textbf{/* does not need to specified a type for this variable */} \par a ="a string"; \par
\textbf{/* it valid to change the type of the variable */} \\ 
\hline
\end{tabular} \\\\
In my project, I used the dynamic typing scheme,which is ,does not need to specify any type of variable and able to assign any type of primitives to a variable.

\subsection{Strong Typing and Weak Typing}
a language is said to be strong typing is that it place restriction in operation where data type can not be intermix.


\begin{tabular}{p{6cm}|p{6cm}}
\hline Strong Typing & Weak Typing  \\ 
\hline a=123; \par 
\textbf{/* a is a number */} \par 
b="123" \par  
\textbf{/* b is a string */} \par 
c=a+b \textbf{/* return type error */}  &  
a = 123; \par 
b = "123"; \par
c = a+b; \par 
\textbf{/* either a will be convert to a string or b will be convert to a number */}

\\ 
\hline 
\end{tabular} 

In my project,I have implemented a weak typing system .I have design an statement call generic expression , which allow different kinds of value to intermix with each other. An expression like "12343" + 1232 -324 can be parse as follow syntax tree.





\section{Problem and resolution in writing BNF/EBNF rules}
\subsubsection{Shift//Reduce Problem}

\subsubsection{Reduce//Reduce Problem}


