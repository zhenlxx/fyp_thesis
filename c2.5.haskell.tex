\chapter{Haskell}
\section{History}
Lambda calculus was introduced by Alonzo Church in the 1930s as part of
an investigation into the foundations of mathematics. [4]
In 1936 Church isolated and published just the portion relevant to computa-
tion, what is now called the untyped lambda calculus. [5]


Following the release of Miranda by in 1985, interest in lazy functional
languages grew: by 1987, more than a dozen non-strict, purely functional
programming languages existed. Of these, Miranda was the most widely
used, but was not in the public domain.
At the conference on Functional Programming Languages and Computer Ar-
chitecture (FPCA ’87) in Portland, Oregon, a meeting was held during which
participants formed a strong consensus that a committee should be formed
to define an open standard for such languages.[20]


The first version of Haskell (”Haskell 1.0”) was defined in 1990. [14] The
latest Haskell version describe in the Haskell report 2010,the implementation
of the latest version Haskell Platform is released in Feb 2011.
To summarize it ,Haskell has evolved continuously in recent years.

Category theory is a general theory that examine and organize mathematical
object like set ,function,function domains Cartesian-set.


\section{Haskell and Category Theory}
Category theory is a general theory that examine and organize mathematical object like set ,function,function domains Cartesian-set.

A Category $C $ in category theory is defined below :
\begin{enumerate}
\item a collection of objects 
\item a collection of arrows (often call morphism) 
\item operations assigning to each arrow $f$ an object $dom\;f$,its domain ,and an object $cod\;f$,its co domain.
\item a composition operator assigning to each pair of arrows $f and g$,with $cod\;f = dom\;g$,a composite arrow $ g \circ f:dom\;f \rightarrow  cod\;g$ , satisfying the following associative law: \\
For any arrow $f: A \rightarrow B,g:B \rightarrow C,and\;h: C\rightarrow D$(with A,B,C and D not necessarily distinct),
$$h\circ (g\circ f) = (h\circ g)\circ f$$
\item for each object A, an identify arrow $id_{a}: A \rightarrow A$ satisfying the following identity law:\\
For any arrow $ f: A \rightarrow B,$ 
$$ id_{a} \circ f = f  \;and\;  f\circ id_{a} = f. $$
\end{enumerate}\cite{pierce_basic_1991}

\section{Algebraic Data Type}
The algebraic data type allow programming to define a data type that has
more than one constructor. For example , a boolean value can be defined as,


\section{Type Classes}


\section{Higher Order Functions}